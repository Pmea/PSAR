\documentclass[a4paper, titlepage, oneside, 12pt]{article}%      autres choix : book  report

\usepackage[utf8]{inputenc}%           gestion des accents (source)
\usepackage[T1]{fontenc}%              gestion des accents (PDF)
\usepackage[francais]{babel}%          gestion du français
\usepackage{textcomp}%                 caractères additionnels
\usepackage{mathtools,  amssymb, amsthm}% packages de l'AMS + mathtools
\usepackage{lmodern}%                  police de caractère
\usepackage{geometry}%                 gestion des marges
\usepackage{graphicx}%                 gestion des images
\usepackage{xcolor}%                   gestion des couleurs
\usepackage{array}%                    gestion améliorée des tableaux
\usepackage{calc}%                     syntaxe naturelle pour les calculs
\usepackage{titlesec}%                 pour les sections
\usepackage{titletoc}%                 pour la table des matières
\usepackage{fancyhdr}%                 pour les en-têtes
\usepackage{titling}%                  pour le titre
\usepackage[framemethod=TikZ]{mdframed}% print frames
\usepackage{caption}%                  for captionof
\usepackage{listings}
\usepackage{enumitem}%                 pour les listes numérotées
\usepackage{microtype}%                améliorations typographiques
\usepackage{csvsimple}%                 convertir un fichier .csv en tableau

\usepackage{hyperref}%                 gestion des hyperliens

\usepackage{titling} %  				  gestion des subtitles 
\newcommand{\subtitle}[1]{%			  definition d'une nouvelle commande sous-titre
  \posttitle{%
    \par\end{center}
    \begin{center}\large#1\end{center}
    \vskip0.5em}%
}                

\hypersetup{%
    pdfborder = {0 0 0}
}

                                    
\title{Projet PSAR :\\ Dispositif Autonome de Synthèse Sonore}
\subtitle{Encadrant : Hugues Genevois \\ Cahier de Charges}

\author{Pierre Mahé \& Adrien Hug-Korda}
\date{\today}
 
 
\begin{document} 
\maketitle 
\tableofcontents

\newpage

\section{Contexte}

\paragraph{}
\textbf{Maîtrise d’œuvre :} Hugues GENEVOIS, Michel RISSE
\vspace{-5mm}
\paragraph{}
\textbf{Maîtrise d’ouvrage :} Pierre MAHÉ, Adrien HUG-KORDA 

\paragraph{}
Michel RISSE, compositeur de Décor sonore, a à son actif de nombreuses création et dispositifs sonore.
Le projet DASS, Dispositif Autonome de Synthèse Sonore, est un nouveau projet de Michel RISSE, compositeur de Décor sonore aillant a à a son actif de nombreuses créations et dispositifs sonores. Ce projet en fait en collaboration avec Hugues GENEVOIS, Chercheur au LAM de l’Institut Jean le Rond d’Alembert.

Le projet consiste a mettre en place un dispositif permettant, à l'aide de capteurs de recueillir des informations sur l'environnement qui l’entour et en émettant des sons, interagir avec cette environnement. 

Dans un premier temps, ce dispositif devrait être utilisé pour une installation en juillet prochain dans le contexte de [nom de l’événement] à [nom du lieux].
Dans un second temps Hugues GENEVOIS et Michel RISSE voudraient réaliser une installation plus importante avec de plusieurs dispositifs avec pour chacun des comportements et des caractéristiques au niveau de la synthèse sonore. 

\section{Le produit}
\subsection{Description de la demande}

\paragraph{}
Le but du projet est de créer un dispositif électroacoustiques portable interagissant avec son environnement.
Ce dispositif doit capter l'environnement qui l'entour a l'aide de capteurs diverses (micro, luminosité, température...). Il devra traiter ces données pour en extraire des informations pertinent sur les changements se produisant autour de lui (chant d'oiseaux, passage d'une personne au alentour...).
Grâce a ses informations, il pourra interagir en synthétisant des signaux acoustiques. \\
Cette synthèse devra être paramétrable, pour permettre au compositeur de modifier l’influence des informations capter sur la synthèse.

\paragraph{}
Une interface utilisateur, permettant au compositeur de simuler des données captés, sera intégrer au dispositif. Cela lui permettra de faire des expérimentations pour sa composition.

\paragraph{}
Dans l'optique de la reproduction de ce dispositif pour des installations plus important. Une procédure devra être pour permettre a un utilisateur de pouvoir reproduire le système, aussi bien matériel que logiciel.

\paragraph{}
Le dispositif devant être portable, le programme doit fonctionner sur une nano-ordinateur Udoo. Et le langage du programme principal est PureData, pour assurer une maintenabilité et une extensionnalité plus facile.



\subsection{Fonctions du produit}
\paragraph{}
L’acquisition des donnée des capteurs devra être traité en temps réel pour permettre une bonne interaction avec son environnement. Les ressources du nano-ordianteur étant limité, l'analyse des données devra être économe en temps de calcule. Ainsi que de la taille des données devant être stocké sur la carte mémoire.\\
La production de son du dispositif devra être une synthèse sonore et non un lecture de sons prêts enregistrés. De plus la synthèse sonore aura pour rôle de souligner les éléments déjà présent dans l’environnement  et on ne faire un imitation de celle-ci.

\paragraph{}
La synthèse paramétrable devra être modifiable par un interface ergonomique sans avoir à brancher d’écran, ni de clavier au nano-ordinateur. Cette interface permet de définir ce que chaque capteur contrôle au niveau de la synthèse.

\paragraph{}
L'interface utilisateur devra permettre a l'utilisateur de fournir, a l'aide d'un fichier ou en traçant une courbe, les valeurs désiré pour chaque capteur pour faire des simulations du comportement du dispositif en condition réel. Certains paramètres variant tres lentement comme la température ou la pression, l'utilisateur pourra également modifier la vitesse de lecture de ses valeurs.

\paragraph{}
La procédure permettant la reproduction du dispositif un tutoriel détaillant toute les actions a faire sera fourni pour le coté logiciel ainsi que les schémas des divers montages pour le coté matériel. En parallèle a ce tutoriel, une image de la carte mémoire du nano-ordinateur sera fourni. Ce qui permettra la duplication rapide pour le même type de carte (même modelé, même version).

\paragraph{}
Une fois le dispositif installé, il faudra pouvoir savoir, sans le toucher, si il fonctionne correctement et si aucun capteur n'est défectueux.


\section{Contraintes}
\subsection{Contraintes materiel/logiciel}
\paragraph{}
Le programme doit fonctionner sur un nano-ordinateur Udoo et doit utiliser majoritairement les possibilités du langage PureData avec possibilité de réaliser des portions de code dans d'autres langages.

\subsection{Contraintes fonctionnement}
\paragraph{}
Le dispositif doit être autonome; l'utilisateur allume le nano-ordineur puis le système se charge et peut doit pouvoir s’exécuter pendant plusieurs heurs (voir plusieurs jours) sans besoin de l’intervention quelconque.

\paragraph{}
De plus il serai envisager, pour une future version, que le dispositif soit autonome énergétiquement en embarquant des panneaux photo-voltaïques.

\subsection{Contraintes d'extensibilité}
\paragraph{}
Le dispositif n'est qu'une première version qui devra être re-utilisable pour des versions plus élaboré du projet, qui permettrai de le rendre autonome énergiquement, ou permettrai de faire communiquer les dispositifs entre eux.\\ Dans cet optique le projet doit être maintenable et extensible le plus facilement possible.

\subsection{Contraintes de Temps}
\paragraph{}
Le produit doit être livrée le 12/05/2015.

\section{Tests de validation}

\subsection{Test d'interaction}
\paragraph{Test\\}
Ce test permet de savoir si le dispositif fonctionne bien.

\paragraph{Hypothèse\\}
Le dispositif est déployé, tous les composants fonctionnent correctement.

\paragraph{Entrée\\}
Néant.

\paragraph{Enchaînement nominal}
\begin {enumerate}
\item L'utilisateur démarre le système. Il attend que le système soit prêts.
\item L'utilisateur fait une suite de plusieurs claquements de main.
\end{enumerate}

\paragraph{Résultat attendu\\}
Le système doit répondre a ses claquements de main en émettant un son en lien (rythme, fréquence) avec l'action de l'utilisateur.

\subsection{Test de l'interface d'utilisation}
\paragraph{Test\\}
Envoi de donnés via l'interface d'utilisation pour simuler une réception particulière d'un capteur.

\paragraph{Hypothèse\\}
Le dispositif est déployé, tous les composants fonctionnent correctement.
Le nano-ordinateur est éteint.

\paragraph{Entrée\\}
nom du fichier \"capteur\_val.dat\" contenant un ensemble de valeurs.\\
Le capteur de température comme le capteur cible.


\paragraph{Enchaînement nominal}
\begin {enumerate}
\item L'utilisateur démarre le système.
\item L'utilisateur envoi a l'aide de l'interface le fichier \"capteur\_val.dat\" et choisi le capteur de température comme capteur correspond ces valeurs.
\item L'utilisateur confirme.
\end{enumerate}

\paragraph{Résultat attendu\\}
Les valeurs on était enregistré par le dispositif. Le dispositif doit générer du son.


\subsection{Test de duplication}

\paragraph{Test\\}
Ce test permet de voir si le tutoriel fourni avec le projet est suffisant pour reproduire le dispositif.
\paragraph{Hypothèse\\}
L'utilisateur dispose de tous le matériel nécessaire pour faire le montage.
De plus il possède un ordinateur sous Linux a ça disposition.

\paragraph{Entrée\\}
Une carte microSD.

\paragraph{Enchaînement nominal}
\begin {enumerate}
\item L'utilisateur fait les montages indiqué sur les schémas fourni avec le projet.
\item L'utilisateur suit la procédure d'installation détaillé dans le tutoriel. Qui lui permet d'installer le logiciel sur le nano-ordinateur.
\item L'utilisateur redémarre le nano-ordinateur.
\end{enumerate}

\paragraph{Résultat attendu\\}
Le dispositif fonctionne, les voyants des capteurs indiquent qu'ils fonctionnes. Le dispositif émet des sons.

\end{document}