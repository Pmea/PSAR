\documentclass[a4paper, titlepage, oneside, 12pt]{article}%      autres choix : book  report

\usepackage[utf8]{inputenc}%           gestion des accents (source)
\usepackage[T1]{fontenc}%              gestion des accents (PDF)
\usepackage[francais]{babel}%          gestion du français
\usepackage{textcomp}%                 caractères additionnels
\usepackage{mathtools,  amssymb, amsthm}% packages de l'AMS + mathtools
\usepackage{lmodern}%                  police de caractère
\usepackage{geometry}%                 gestion des marges
\usepackage{graphicx}%                 gestion des images
\usepackage{xcolor}%                   gestion des couleurs
\usepackage{array}%                    gestion améliorée des tableaux
\usepackage{calc}%                     syntaxe naturelle pour les calculs
\usepackage{titlesec}%                 pour les sections
\usepackage{titletoc}%                 pour la table des matières
\usepackage{fancyhdr}%                 pour les en-têtes
\usepackage{titling}%                  pour le titre
\usepackage[framemethod=TikZ]{mdframed}% print frames
\usepackage{caption}%                  for captionof
\usepackage{listings}
\usepackage{enumitem}%                 pour les listes numérotées
\usepackage{microtype}%                améliorations typographiques
\usepackage{csvsimple}%                 convertir un fichier .csv en tableau
\usepackage{url}
\usepackage{hyperref}%                 gestion des hyperliens

\usepackage{titling} %  				  gestion des subtitles 
\newcommand{\subtitle}[1]{%			  definition d'une nouvelle commande sous-titre
  \posttitle{%
    \par\end{center}
    \begin{center}\large#1\end{center}
    \vskip0.5em}%
}                
\lstset{language=c++}
\definecolor{codegreen}{rgb}{0,0.6,0}
\definecolor{codegray}{rgb}{0.5,0.5,0.5}
\definecolor{codepurple}{rgb}{0.58,0,0.82}
\definecolor{backcolour}{rgb}{0.95,0.95,0.92}
 
\lstdefinestyle{mystyle}{
    backgroundcolor=\color{backcolour},   
    commentstyle=\color{codegreen},
    keywordstyle=\color{magenta},
    numberstyle=\tiny\color{codegray},
    stringstyle=\color{codepurple},
    basicstyle=\footnotesize,
    breakatwhitespace=false,         
    breaklines=true,                 
    captionpos=b,                    
    keepspaces=true,                 
    numbers=left,                    
    numbersep=5pt,
    otherkeywords={uint16_t},                  
    showspaces=false,                
    showstringspaces=false,
    inputencoding=latin1,
    showtabs=false,                  
    tabsize=2
}
\lstset{style=mystyle}

\hypersetup{%
    pdfborder = {0 0 0}
}


                                    
\title{Rapport du Projet PSAR :\\ Dispositif Autonome de Synthèse Sonore}
\subtitle{Encadrant : Hugues Genevois \\ Cahier de Charges}

\author{Pierre Mahé}
\date{\today}
 
 
\begin{document} 
\maketitle 
\tableofcontents

\newpage
\section{Introduction}
\subsection{Presentation du projet}

\subsection{La Carte Udoo}

\subsection{Pure Data}
\subsubsection{Externals}

\subsection{Structure du projet}

\newpage
\section{Traitement audio}
\subsection{Aquisition audio}

\subsection{Traitement bas niveau}

\subsubsection{Filtrage}
\subsubsection{Détecteur de notes}
\subsubsection{Bandes de fréquences}

\subsection{Extraction des méta-données}
\subsubsection{Mélodie et rythme}
\subsubsection{Pattern minimal}

\newpage
\section{Récuperation de l'environnement}
\subsection{Dispositif et capteurs}

\subsection{Communication inter plate-forme}

\subsection{External pour la communication}

\newpage
\section{Synthèse musical}
% changement de boite pour chaque execution
\subsection{Modele physique au longterme}

\subsection{Synthèse implémenté}

\newpage
\section{Interface Utilisateur}
\subsection{Premiere idée d'implementation}
% pouvoir s'abstrere de pure data
% avec étude du protocole netsend 

\subsection{Envoie des données capteur}

\subsection{Envoie d'objet Pure data}

\newpage
\section{Tutoriel et Documentation}
\subsection{Écriture documentation Pure data}
\subsection{Écriture du Tutoriel d'installation}
% sauvegarde de carte et tuto

\newpage
\section{Pour aller plus loin}
\subsection{Tests en environnement reel}
\subsection{Tests énergétiques}
\subsection{Serveur distant}

\newpage
\section{Bibliographie}
\nocite{*}
\bibliographystyle{plain}
\bibliography{biblio-projet-DASS}
%\addcontentsline{toc}{chapter}{Bibliographie}

\end{document}